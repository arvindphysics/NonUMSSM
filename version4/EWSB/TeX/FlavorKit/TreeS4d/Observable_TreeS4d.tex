\documentclass[A4,landscape]{article} 
\usepackage{amsmath}
\usepackage[T1]{fontenc}
\usepackage{amssymb}
\usepackage{feynmp}
\usepackage{hyperref}
\usepackage{longtable}
\DeclareGraphicsRule{*}{mps}{*}{}
\graphicspath{{./Diagrams/}}
\textwidth 25cm
\evensidemargin -1.0cm
\oddsidemargin -1.0cm
\begin{document}
\title{Analytical expressions for the form factors of TreeS4d\\ in the $Generelized Family Non-Universal U(1)$ extended MSSM } 
 \author{SARAH 4.14.3} 
 \maketitle 
 \vspace{10cm} 
This file was automatically generated by SARAH version 4.14.3.  \\ 
References: {\bf arXiv: 1309.7223 }, {\bf Comput.Phys.Commun.184:1792-1809,2011 (1207.0906) }, {\bf Comput.Phys.Commun.182:808-833,2011 (1002.0840) }, {\bf Comput.Phys.Commun.181:1077-1086,2010 (0909.2863) }, {\bf arXiv: 0806.0538 } \\ 
Package Homepage: projects.hepforge.org/sarah/ \\ 
by {\bf Florian Staub, florian.staub@kit.edu} 
 \pagebreak 
 \tableofcontents 
 \pagebreak 
\section{External states: ${d_{{i}}, \bar{d}_{{j}}, d_{{k}}, \bar{d}_{{l}}}$} 
\subsection{Tree contributions, Propagator: $h$} 

\begin{align} 
  TSO4dSLL= & \Gamma^{\bar{d}d h ,L}_{j, i, p} \Gamma^{\bar{d}d h ,L}_{l, k, p} \frac{1}{m^2_{h_{{p}}}} \\ 
  TSO4dSRR= & \Gamma^{\bar{d}d h ,R}_{j, i, p} \Gamma^{\bar{d}d h ,R}_{l, k, p} \frac{1}{m^2_{h_{{p}}}} \\ 
  TSO4dSRL= & \Gamma^{\bar{d}d h ,R}_{j, i, p} \Gamma^{\bar{d}d h ,L}_{l, k, p} \frac{1}{m^2_{h_{{p}}}} \\ 
  TSO4dSLR= & \Gamma^{\bar{d}d h ,L}_{j, i, p} \Gamma^{\bar{d}d h ,R}_{l, k, p} \frac{1}{m^2_{h_{{p}}}} \\ 
  TSO4dVRR= & 0 \\ 
  TSO4dVLL= & 0 \\ 
  TSO4dVRL= & 0 \\ 
  TSO4dVLR= & 0 \\ 
  TSO4dTLL= & 0 \\ 
  TSO4dTLR= & 0 \\ 
  TSO4dTRL= & 0 \\ 
  TSO4dTRR= & 0 \\ 
\end{align} 
\begin{align} 
  TSO4dSLL= & -(\Gamma^{\bar{d}d h ,L}_{l, i, p} \Gamma^{\bar{d}d h ,L}_{j, k, p} \frac{1}{m^2_{h_{{p}}}})/2 \\ 
  TSO4dSRR= & -(\Gamma^{\bar{d}d h ,R}_{l, i, p} \Gamma^{\bar{d}d h ,R}_{j, k, p} \frac{1}{m^2_{h_{{p}}}})/2 \\ 
  TSO4dSRL= & 0 \\ 
  TSO4dSLR= & 0 \\ 
  TSO4dVRR= & 0 \\ 
  TSO4dVLL= & 0 \\ 
  TSO4dVRL= & -(\Gamma^{\bar{d}d h ,R}_{l, i, p} \Gamma^{\bar{d}d h ,L}_{j, k, p} \frac{1}{m^2_{h_{{p}}}})/2 \\ 
  TSO4dVLR= & -(\Gamma^{\bar{d}d h ,L}_{l, i, p} \Gamma^{\bar{d}d h ,R}_{j, k, p} \frac{1}{m^2_{h_{{p}}}})/2 \\ 
  TSO4dTLL= & (\Gamma^{\bar{d}d h ,L}_{l, i, p} \Gamma^{\bar{d}d h ,L}_{j, k, p} \frac{1}{m^2_{h_{{p}}}})/8 \\ 
  TSO4dTLR= & 0 \\ 
  TSO4dTRL= & 0 \\ 
  TSO4dTRR= & (\Gamma^{\bar{d}d h ,R}_{l, i, p} \Gamma^{\bar{d}d h ,R}_{j, k, p} \frac{1}{m^2_{h_{{p}}}})/8 \\ 
\end{align} 
\subsection{Tree contributions, Propagator: $A^0$} 

\begin{align} 
  TSO4dSLL= & \Gamma^{\bar{d}d A^0 ,L}_{j, i, p} \Gamma^{\bar{d}d A^0 ,L}_{l, k, p} \frac{1}{m^2_{A^0_{{p}}}} \\ 
  TSO4dSRR= & \Gamma^{\bar{d}d A^0 ,R}_{j, i, p} \Gamma^{\bar{d}d A^0 ,R}_{l, k, p} \frac{1}{m^2_{A^0_{{p}}}} \\ 
  TSO4dSRL= & \Gamma^{\bar{d}d A^0 ,R}_{j, i, p} \Gamma^{\bar{d}d A^0 ,L}_{l, k, p} \frac{1}{m^2_{A^0_{{p}}}} \\ 
  TSO4dSLR= & \Gamma^{\bar{d}d A^0 ,L}_{j, i, p} \Gamma^{\bar{d}d A^0 ,R}_{l, k, p} \frac{1}{m^2_{A^0_{{p}}}} \\ 
  TSO4dVRR= & 0 \\ 
  TSO4dVLL= & 0 \\ 
  TSO4dVRL= & 0 \\ 
  TSO4dVLR= & 0 \\ 
  TSO4dTLL= & 0 \\ 
  TSO4dTLR= & 0 \\ 
  TSO4dTRL= & 0 \\ 
  TSO4dTRR= & 0 \\ 
\end{align} 
\begin{align} 
  TSO4dSLL= & -(\Gamma^{\bar{d}d A^0 ,L}_{l, i, p} \Gamma^{\bar{d}d A^0 ,L}_{j, k, p} \frac{1}{m^2_{A^0_{{p}}}})/2 \\ 
  TSO4dSRR= & -(\Gamma^{\bar{d}d A^0 ,R}_{l, i, p} \Gamma^{\bar{d}d A^0 ,R}_{j, k, p} \frac{1}{m^2_{A^0_{{p}}}})/2 \\ 
  TSO4dSRL= & 0 \\ 
  TSO4dSLR= & 0 \\ 
  TSO4dVRR= & 0 \\ 
  TSO4dVLL= & 0 \\ 
  TSO4dVRL= & -(\Gamma^{\bar{d}d A^0 ,R}_{l, i, p} \Gamma^{\bar{d}d A^0 ,L}_{j, k, p} \frac{1}{m^2_{A^0_{{p}}}})/2 \\ 
  TSO4dVLR= & -(\Gamma^{\bar{d}d A^0 ,L}_{l, i, p} \Gamma^{\bar{d}d A^0 ,R}_{j, k, p} \frac{1}{m^2_{A^0_{{p}}}})/2 \\ 
  TSO4dTLL= & (\Gamma^{\bar{d}d A^0 ,L}_{l, i, p} \Gamma^{\bar{d}d A^0 ,L}_{j, k, p} \frac{1}{m^2_{A^0_{{p}}}})/8 \\ 
  TSO4dTLR= & 0 \\ 
  TSO4dTRL= & 0 \\ 
  TSO4dTRR= & (\Gamma^{\bar{d}d A^0 ,R}_{l, i, p} \Gamma^{\bar{d}d A^0 ,R}_{j, k, p} \frac{1}{m^2_{A^0_{{p}}}})/8 \\ 
\end{align} 
\subsection{Box contributions} 

\end{document}
